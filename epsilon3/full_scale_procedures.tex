\launchchecklist{    1.1 & Restricting the parachute, shroud lines, and shock cord can result in the vehicle falling without any method to slow down, leading to a failed recovery.\lcwarning & Gloves & CE & \checkbox Remove tape on drogue parachute, shroud lines, and shock cord.\\\hline    1.2 &  & Gloves & CE & \checkbox Verify the parachute, shroud lines, and shock cord are free of damage.\\\hline    1.3 & Improper storage of the drogue parachute shroud lines can result in the parachute failing to open fully after deployment. This could produce unexpectedly high descent velocities and lead to a failed recovery.\lcwarning & Gloves & CE & \checkbox Untangle parachute shroud lines\\\hline    1.4 & Improper folding can result in the parachute failing to eject or open fully after deployment. This could produce unexpectedly high descent velocities and lead to a failed recovery.\lcwarning & Gloves & CE & \checkbox Z-fold the parachute\\\hline    1.5 & If Nomex does not fully cover the parachute, black powder combustion may burn holes in the parachute. This could produce unexpectedly high descent velocities and lead to a failed recovery\lcwarning & Gloves & CE & \checkbox Wrap parachute with the small square of nomex\\\hline    1.6 &  & Gloves & CE & \checkbox Wrap the parachute in a rubber band to maintain shape.\\\hline    1.7 & If Nomex is not attached at the base of the shroud lines, it may lift up along the shroud lines and reef the parachute resulting in high descent velocities and a failed recovery.\lcwarning & Gloves & CE & \checkbox Connect the parachute swivel to the nomex with a quick link.\\\hline    1.8 & Improperly attached quick link may result in the drogue parachute separating from the launch vehicle and cause failed recovery\lcwarning & Gloves & CE & \checkbox Verify that the drogue shock cord has a loop for parachute attachment.\\\hline    1.9 & Nomex should cover the majority of the area to prevent black powder combustion from reaching parts of the parachute\lcwarning & Gloves & CE & \checkbox Cover the 3 ft closest to the avionics bay with a shock cord protector.\\\hline    1.10 & If Nomex slips, the region holding the parachute may be exposed to black powder ccombustion\lcwarning & Gloves & CE & \checkbox Apply duct tape to both ends of the shock cord protector.\\\hline    1.11 & Bundles that are too tight will inhibit shock cord from fully expanding; bundles that are too sparse will not be effective in absorbing force\lcwarning & Gloves & CE & \checkbox Form bundles of shock cord every 12" along the length of the shock cord.\\\hline    1.12 &  & Gloves & CE & \checkbox Attach the drogue parachute to the shock cord.\\\hline    1.13 &  & Gloves & CE & \checkbox Attach a quick link to the forward end of the shock cord.\\\hline    1.14 &  & Gloves & CE & \checkbox Attach a quick link to the aft end of the shock cord.\\\hline    1.15 &  & Gloves & CE, LVSTL, SO & \checkbox Set aside the completed parachute and recovery harness.\\\hline}{L-1 Drogue Parachute Preparation}{tab:safety:l-1_drogue_parachute_preparation}

\launchchecklist{    2.1 & Restricting the main parachute, shroud lines, and shock cord can result in the vehicle falling without any method to slow down, leading to a failed recovery.\lcwarning & Gloves & CE & \checkbox Remove tape on main parachute, shroud lines, and shock cord.\\\hline    2.2 &  & Gloves & CE & \checkbox Verify the parachute, shroud lines, and shock cord are free of damage.\\\hline    2.3 & Improper storage of the parachute shroud lines can result in the parachute failing to open fully after deployment. This could produce unexpectedly high descent velocities and lead to a failed recovery.\lcwarning & Gloves & CE & \checkbox Untangle parachute shroud lines\\\hline    2.4 & Improper folding can result in the parachute failing to eject or open fully after deployment. This could produce unexpectedly high descent velocities and lead to a failed recovery.\lcwarning & Gloves & CE & \checkbox Z-fold the parachute\\\hline    2.5 & If Nomex does not fully cover the parachute, black powder combustion may burn holes in the parachute. This could produce unexpectedly high descent velocities and lead to a failed recovery\lcwarning & Gloves & CE & \checkbox Wrap parachute with the small square of nomex\\\hline    2.6 &  & Gloves & CE & \checkbox Wrap the parachute in a rubber band to maintain shape.\\\hline    2.7 & If Nomex is not attached at the base of the shroud lines, it may lift up along the shroud lines and reef the parachute resulting in high descent velocities and a failed recovery.\lcwarning & Gloves & CE & \checkbox Connect the parachute swivel to the nomex with a quick link.\\\hline    2.8 & Improperly attached quick link may result in the drogue parachute separating from the launch vehicle and cause failed recovery\lcwarning & Gloves & CE & \checkbox Verify that the drogue shock cord has a loop for parachute attachment.\\\hline    2.9 & Nomex should cover the majority of the area to prevent black powder combustion from reaching parts of the parachute\lcwarning & Gloves & CE & \checkbox Cover the 3 ft closest to the avionics bay with a shock cord protector.\\\hline    2.10 & If Nomex slips, the region holding the parachute may be exposed to black powder ccombustion\lcwarning & Gloves & CE & \checkbox Apply duct tape to both ends of the shock cord protector.\\\hline    2.11 & Bundles that are too tight will inhibit shock cord from fully expanding; bundles that are too sparse will not be effective in absorbing force\lcwarning & Gloves & CE & \checkbox Form bundles of shock cord every 12" along the length of the shock cord.\\\hline    2.12 &  & Gloves & CE & \checkbox Attach the main parachute to the shock cord.\\\hline    2.13 &  & Gloves & CE & \checkbox Attach a quick link to the forward end of the shock cord.\\\hline    2.14 &  & Gloves & CE & \checkbox Attach a quick link to the aft end of the shock cord.\\\hline    2.15 &  & Gloves & CE, LVSTL, SO & \checkbox Set aside the completed parachute and recovery harness.\\\hline}{L-1 Main Parachute Preparation}{tab:safety:l-1_main_parachute_preparation}

\launchchecklist{    3.1 &  & Gloves & CE & \checkbox Gather the following items:\\\hline    3.2 &  & Gloves & CE & \checkbox The following steps shall be done by the team mentor.\\\hline    3.3 &  & Gloves & CE & \checkbox Construct a 3.5g charge and label it "MAIN -- PRIMARY"\\\hline    3.4 &  & Gloves & CE & \checkbox Construct a 4.0g charge and label it "MAIN -- BACKUP"\\\hline    3.5 &  & Gloves & CE & \checkbox Construct a 3.5g charge and label it "DROGUE -- PRIMARY"\\\hline    3.6 &  & Gloves & CE & \checkbox Construct a 4.0g charge and label it "DROGUE -- BACKUP"\\\hline    3.7 &  & Gloves & CE, LVSTL, SO & \checkbox Place all charges in antistatic bag.\\\hline}{L-1 Ejection Charge Preparation}{tab:safety:l-1_ejection_charge_preparation}

\launchchecklist{    4.1 &  & Gloves & CE & \checkbox Charge LiPo battery.\\\hline    4.2 &  & Gloves & CE & \checkbox Measure and record battery voltage\\\hline    4.3 &  & Gloves & CE & \checkbox Power on STEMCRaFT capsule by removing the RBF tag\\\hline    4.4 &  & Gloves & CE & \checkbox Power on team transceiver\\\hline    4.5 &  & Gloves & CE & \checkbox Power on team LoRa radio\\\hline    4.6 &  & Gloves & CE & \checkbox Initiate checkout sequence\\\hline    4.7 &  & Gloves & CE & \checkbox Slide STEMCRaFT Capsule onto threaded rod.\\\hline    4.8 &  & Gloves & CE & \checkbox Secure the bottom of the capsule with a 1/4-20 nut.\\\hline    4.9 &  & Gloves & CE & \checkbox Something about cameras???\\\hline    4.10 &  & Gloves & CE & \checkbox Slide the nose cone bulkhead onto the threaded rod.\\\hline    4.11 &  & Gloves & CE & \checkbox Secure the nose cone bulkhead to the threaded rod\\\hline    4.12 &  & Gloves & CE & \checkbox Verify the nose cone bulkhead is secured.\\\hline    4.13 &  & Gloves & PESTL, PMSTL, CE, SO & \checkbox Set the STEMCRaFT Payload aside to await vehicle integration.\\\hline}{L-1 STEMCRaFT Payload Preparation}{tab:safety:l-1_stemcraft_payload_preparation}

\launchchecklist{    5.1 &  & Gloves & CE & \checkbox Charge LiPo battery completely\\\hline    5.2 &  & Gloves & CE & \checkbox Measure and record the LiPo voltage\\\hline    5.3 &  & Gloves & CE & \checkbox Power on ADS payload by removing RBF tag\\\hline    5.4 &  & Gloves & CE & \checkbox Calibrate ADS payload\\\hline    5.5 &  & Gloves & CE & \checkbox Power off ADS payload\\\hline    5.6 &  & Gloves & CE & \checkbox Verify forward u-bolt is secured.\\\hline    5.7 &  & Gloves & CE & \checkbox Verify aft stud is secured.\\\hline    5.8 &  & Gloves & CE & \checkbox Verify forward and aft threaded nuts are secured.\\\hline    5.9 &  & Gloves & PMSTL, PESTL, CE, SO & \checkbox Set the ADS payload aside to await vehicle integration.\\\hline}{L-1 ADS Payload Preparation}{tab:safety:l-1_ads_payload_preparation}

\launchchecklist{    6.1 &  & Gloves & CE & \checkbox Inspect fins and verify that they are undamaged and free of cracks.\\\hline    6.2 &  & Gloves & CE & \checkbox Slide each fin axially until the locator pins are in place.\\\hline    6.3 &  & Gloves & CE & \checkbox Install the thrust plate on top of the fins so that the rear notches align using fin clamps\\\hline    6.4 &  & Gloves & CE & \checkbox Attach the thrust plate to the four fins using 1/4-20'' x 0.5'' bolts\\\hline    6.5 &  & Gloves & CE & \checkbox Verify the motor mount tube is secured in place.\\\hline    6.6 &  & Gloves & CE & \checkbox Verify that the fins cannot move radially or axially.\\\hline    6.7 &  & Gloves & CE & \checkbox Verify that the bolts holding the fin clamps in place are fully tightened \\\hline    6.8 &  & Gloves & CE & \checkbox Verify that the rail buttons are fully tightened.\\\hline    6.9 &  & Gloves & CE & \checkbox Verify that the following are free of cracks and unintended holes. Airframe. Couplers. Bulkheads.\\\hline    6.10 &  & Gloves & LVSTL, CE, SO & \checkbox Sign-off for vehicle integrity. \\\hline}{L-1 Vehicle Inspection}{tab:safety:l-1_vehicle_inspection}

\launchchecklist{    7.1 &  & Gloves & CE & \checkbox Locate the following: Nose cone subassembly. Main parachute recovery harness. Forward airframe.\\\hline    7.2 &  & Gloves & CE & \checkbox Pass the forward end of the recovery harness through the forward airframe.\\\hline    7.3 &  & Gloves & CE & \checkbox Connect the forward end of the recovery harness to the nose cone bulkhead.\\\hline    7.4 &  & Gloves & CE & \checkbox Verify the quick link is tight by checking with a teammate. \\\hline    7.5 &  & Gloves & CE & \checkbox Slide the nose cone shoulder into the forward airframe. \\\hline    7.6 &  & Gloves & CE & \checkbox Connect the nose cone to the forward airframe with 4x 4-40 shear pins.\\\hline    7.7 &  & Gloves & CE & \checkbox Push the main parachute into the forward airframe.\\\hline    7.8 & Push in small sections to prevent tangling. \lcwarning & Gloves & CE & \checkbox Push the recovery harness into the forward airframe \\\hline    7.9 &  & Gloves & CE & \checkbox Insert 2x large handfuls of dog barf (cellulose insullation)\\\hline    7.10 &  & Gloves & CE & \checkbox Cap off the forward airframe.\\\hline    7.11 &  & Gloves & CE & \checkbox Locate the following: Aft airframe. Drogue parachute recovery harness. ADS subassembly. Booster subassembly.\\\hline    7.12 &  & Gloves & CE & \checkbox Rotate the ADS mechanism to the clocked orientation.\\\hline    7.13 &  & Gloves & CE & \checkbox Connect the ADS coupler to the booster airframe with 4x 4-40 shear pins.\\\hline    7.14 &  & Gloves & CE & \checkbox Pass the aft end of the recovery harness through the aft airframe front to back.\\\hline    7.15 &  & Gloves & CE & \checkbox Connect the recovery harness to the ADS forward bulkhead.\\\hline    7.16 &  & Gloves & CE & \checkbox Verify the quick link is tight by checking with a teammate. \\\hline    7.17 &  & Gloves & CE & \checkbox Slide the ADS forward coupler into the aft airframe.\\\hline    7.18 &  & Gloves & CE & \checkbox Connect the aft airframe to the ADS coupler with 4x 4-40 shear pins.\\\hline    7.19 &  & Gloves & CE & \checkbox Push the recovery harness into the aft airframe.\\\hline    7.20 &  & Gloves & CE & \checkbox Push the drogue parachute into the aft airframe.\\\hline    7.21 &  & Gloves & CE & \checkbox Push the remaining recovery harness into the aft airframe.\\\hline    7.22 &  & Gloves & CE & \checkbox Insert 2x large handfuls of dog barf (cellulose insullation)\\\hline    7.23 &  & Gloves & CE & \checkbox Cap off the aft airframe.\\\hline    7.24 &  & Gloves & CE, LVSTL, SO & \checkbox Sign-off for vehicle readiness.\\\hline}{L-1 Preliminary Vehicle Integration}{tab:safety:l-1_preliminary_vehicle_integration}

\launchchecklist{    8.1 &  & Gloves & CE & \checkbox Connect jumper wires to 4x e-match passthroughs on the avionics stack.\\\hline    8.2 &  & Gloves & CE & \checkbox Place the avionics stack in a vertical flight-like configuration.\\\hline    8.3 &  & Gloves & CE & \checkbox Power on the backup computer.\\\hline    8.4 &  & Gloves & CE & \checkbox Verify 3x beeps to indicate nominal continuity.\\\hline    8.5 &  & Gloves & CE & \checkbox Power off the backup computer.\\\hline    8.6 &  & Gloves & CE & \checkbox Power on the primary computer.\\\hline    8.7 &  & Gloves & CE & \checkbox Verify 3x beeps to indicate nominal continuity.\\\hline    8.8 &  & Gloves & CE & \checkbox Verify signal acquisition by TeleBT ground station.\\\hline    8.9 &  & Gloves & CE & \checkbox Power off primary computer.\\\hline    8.10 &  & Gloves & CE & \checkbox Remove 4x jumper wires.\\\hline    8.11 &  & Gloves & CE & \checkbox Gather the following: Forward vehicle subassembly. Avionics stack. 4x black powder charges. Aft vehicle subassembly.\\\hline    8.12 & The following steps involve black powder explosives and shall only be completed by the team mentor.\lccritical & Gloves & CE & \checkbox The following steps shall be done by the team mentor.\\\hline    8.13 &  & Gloves & CE & \checkbox Verify the team mentor and all team members are wearing safety glasses.\\\hline    8.14 &  & Gloves & CE & \checkbox Uncap the forward airframe. \\\hline    8.15 &  & Gloves & CE & \checkbox Connect the main recovery harness to the forward avionics bulkhead.\\\hline    8.16 &  & Gloves & CE & \checkbox Verify the quick link is tight by checking with a friend.\\\hline    8.17 &  & Gloves & CE & \checkbox Connect the "MAIN -- PRIMARY" charge to the correct terminal.\\\hline    8.18 &  & Gloves & CE & \checkbox Connect the "MAIN -- BACKUP" charge to the correct terminal.\\\hline    8.19 &  & Gloves & CE & \checkbox Slide the forward airframe onto the forward avionics coupler.\\\hline    8.20 &  & Gloves & CE & \checkbox Connect the sections with 4x 10-32 rivets.\\\hline    8.21 &  & Gloves & CE & \checkbox Uncap the aft airframe.\\\hline    8.22 &  & Gloves & CE & \checkbox Connect the main recovery harness to the forward avionics bulkhead.\\\hline    8.23 &  & Gloves & CE & \checkbox Verify the quick link is tight by checking with a friend.\\\hline    8.24 &  & Gloves & CE & \checkbox Connect the "DROGUE -- PRIMARY" charge to the correct terminal.\\\hline    8.25 &  & Gloves & CE & \checkbox Connect the "DROGUE -- BACKUP" charge to the correct terminal.\\\hline    8.26 &  & Gloves & CE & \checkbox Slide the avionics coupler into the aft airframe.\\\hline    8.27 &  & Gloves & CE & \checkbox Connect the sections with 4x 10-32 rivets.\\\hline    8.28 &  & Gloves & CE & \checkbox Remove motor retainer.\\\hline    8.29 &  & Gloves & CE & \checkbox Slide motor until forward closer contacts ADS stud.\\\hline    8.30 &  & Gloves & CE & \checkbox Rotate motor casing clockwise until aft closure bottoms out on retainer.\\\hline    8.31 &  & Gloves & CE & \checkbox Fasten the motor retainer.\\\hline    8.32 &  & Gloves & CE & \checkbox Verify the motor retainer is tight by checking with a friend.\\\hline    8.33 &  & Gloves & CE & \checkbox Fill out flight card and give it to the PM.\\\hline    8.34 &  & Gloves & CE, PM, SO & \checkbox Sign-off for flight readiness.\\\hline}{L-0 Final Vehicle Integration}{tab:safety:l-0_final_vehicle_integration}

\launchchecklist{    9.1 &  & Gloves & CE & \checkbox Place the launch rail in a horizontal position.\\\hline    9.2 & One person should coordinate the effort. Bending or twisting the vehicle while constrained to the rail can damage the rial buttons.\lcwarning & Gloves & CE & \checkbox Slide the vehicle onto the launch rail. \\\hline    9.3 &  & Gloves & CE & \checkbox Raise the rail to vertical and verify that it is locked.\\\hline    9.4 &  & Gloves & CE & \checkbox Arm the backup altimeter.\\\hline    9.5 &  & Gloves & CE & \checkbox Verify three continuity beeps from the backup altimeter. \\\hline    9.6 &  & Gloves & CE & \checkbox Arm the primary altimeter.\\\hline    9.7 &  & Gloves & CE & \checkbox Verify 3x beeps from the primary altimeter. \\\hline    9.8 &  & Gloves & CE & \checkbox Verify signal acquisition from primary altimeter ground station. \\\hline    9.9 &  & Gloves & CE & \checkbox Remove the motor cap.\\\hline    9.10 &  & Gloves & CE & \checkbox Insert the igniter into the motor.\\\hline    9.11 &  & Gloves & CE & \checkbox Re-attach motor cap to hold the igniter in place.\\\hline    9.12 &  & Gloves & CE & \checkbox Touch alligator clips to check for stray current.\\\hline    9.13 &  & Gloves & CE & \checkbox Connect alligator clips to the igniter wires.\\\hline    9.14 & Clear all personnel away from the pad. Continuity checks have the potential to prematurely fire the rocket motor.\lccritical & Gloves & CE & \checkbox Check igniter for continuity.\\\hline    9.15 &  & Gloves & CE & \checkbox Clear the pad area and await RSO and LCO launch approval.\\\hline    9.16 &  & Gloves & CE & \checkbox Launch.\\\hline    9.17 & MISSING STEPS\lccritical & Gloves & CE & \checkbox Need steps for troubleshooting maybe as separate procedures.\\\hline}{Pad Operations}{tab:safety:pad_operations}

\launchchecklist{    10.1 &  & Gloves & CE & \checkbox Locate the vehicle using visual cues or the GPS locator.\\\hline    10.2 &  & Gloves & CE & \checkbox Check surroundings around vehicle to ensure no part has landed in a hazardous environment\\\hline    10.3 &  & Gloves & CE & \checkbox Take ample photos of every section of the vehicle.\\\hline    10.4 &  & Gloves & CE & \checkbox Disarm both altimeters.\\\hline    10.5 &  & Gloves & CE & \checkbox Access any encloses charge locations and verify detonation of all charges.\\\hline    10.6 &  & Gloves & CE & \checkbox Record observed damage to the vehicle and recovery system.\\\hline    10.7 &  & Gloves & CE & \checkbox Disconnect the primary and drogue recovery harnesses. \\\hline    10.8 &  & Gloves & CE & \checkbox Disconnect and bundle the primary and drogue parachutes.\\\hline    10.9 &  & Gloves & CE & \checkbox Transport all sections of the vehicle back to base.\\\hline}{Recovery Operations}{tab:safety:recovery_operations}

\launchchecklist{    11.1 &  & Gloves & CE & \checkbox Disconnect all vehicle sections and recovery harness hardware.\\\hline    11.2 &  & Gloves & CE & \checkbox Clean the inside of the vehicle airframes using isopropyl alcohol.\\\hline    11.3 &  & Gloves & CE & \checkbox Disassemble the avionics bay and remove ungrounded electrical connections.\\\hline    11.4 &  & Gloves & CE & \checkbox Clean the avionics bay bulkheads using isopropyl alcohol.\\\hline    11.5 & Isopropyl alcohol will degrade paint if present on the vehicle.\lccritical & Gloves & CE & \checkbox Clean the outside of the vehicle with a damp cloth.\\\hline}{L+1 Vehicle Cleaning}{tab:safety:l+1_vehicle_cleaning}

